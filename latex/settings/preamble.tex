% Компилятор: XeLaTeX
\documentclass[14pt, a4paper]{extreport}

% Кодировка и языки
\usepackage{polyglossia}
\setdefaultlanguage{russian}
\setotherlanguage{english}

% Шрифты
\usepackage{fontspec}
\setmainfont{Liberation Serif}
\setsansfont{Liberation Sans}
\setmonofont{Liberation Mono}

% Математические шрифты
\usepackage{unicode-math}
\setmathfont{Libertinus Math}

% Математические пакеты
\usepackage{amsmath}
\usepackage{amsthm}
\usepackage{mathtools}

% Геометрия страницы по ГОСТ 7.32-2017
\usepackage[
    left=3cm,
    right=1cm,
    top=2cm,
    bottom=2cm
]{geometry}

\setlength{\emergencystretch}{2em} % or 3em, adjust as needed

% Полуторный интервал по ГОСТ
\usepackage{setspace}
\onehalfspacing

% Абзацный отступ 1.25 см по ГОСТ
\usepackage{indentfirst}
\setlength{\parindent}{1.25cm}

% Для включения PDF (титульный лист)
\usepackage{pdfpages}

% Заголовки по ГОСТ
\usepackage{titlesec}

% Главы (chapters)
\titleformat{\chapter}
    {\normalfont\bfseries\centering}
    {\thechapter}{1em}{}
\titlespacing*{\chapter}{0pt}{0pt}{*2}

% Разделы (sections)
\titleformat{\section}
    {\normalfont\bfseries}
    {\thesection}{1em}{}
\titlespacing*{\section}{\parindent}{*2}{*1}

% Подразделы (subsections)
\titleformat{\subsection}
    {\normalfont\bfseries}
    {\thesubsection}{1em}{}
\titlespacing*{\subsection}{\parindent}{*2}{*1}

% Нумерация по ГОСТ
\renewcommand{\thechapter}{\arabic{chapter}}
\renewcommand{\thesection}{\thechapter.\arabic{section}}
\renewcommand{\thesubsection}{\thesection.\arabic{subsection}}

% Содержание
\usepackage{tocloft}
\renewcommand{\cftchapleader}{\cftdotfill{\cftdotsep}}
\renewcommand{\cftsecleader}{\cftdotfill{\cftdotsep}}
\setlength{\cftbeforechapskip}{0pt}
\setlength{\cftbeforesecskip}{0pt}

% Графика
\usepackage{graphicx}
\usepackage{float}
\graphicspath{{images/}{figures/}}

% Подписи к рисункам и таблицам по ГОСТ
\usepackage{caption}
\DeclareCaptionLabelSeparator{defffis}{ -- }
\captionsetup{
    labelsep=defffis,
    font=small,
    justification=centering
}
\renewcommand{\thefigure}{\thechapter.\arabic{figure}}
\renewcommand{\thetable}{\thechapter.\arabic{table}}

% Таблицы
\usepackage{booktabs}
\usepackage{longtable}
\usepackage{multirow}
\usepackage{array}
\hbadness=10000    % hides underfull hbox warnings
\vbadness=10000    % hides underfull vbox warnings  
% Списки
\usepackage{enumitem}
\setlist{nosep, leftmargin=\parindent}

% Библиография
% \usepackage[square,numbers,sort&compress]{natbib}
% \bibliographystyle{unsrt}
% \usepackage[backend=biber, style=authoryear]{biblatex} % or your preferred style
% \addbibresource{biblio/bibliography.bib} % папка biblio, файл biblio.bib
\usepackage[
  backend=biber,
  style=gost-numeric,      % or gost-authoryear, gost-footnote, etc.
  language=auto,
  autolang=other
]{biblatex}

\addbibresource{biblio/bibliography.bib} % path relative to main .tex
% Ссылки
\usepackage{hyperref}
\hypersetup{
    colorlinks=true,
    linkcolor=black,
    filecolor=black,
    urlcolor=blue,
    citecolor=black,
    pdftitle={Магистерская диссертация},
    pdfauthor={Ваше имя}
}

% Листинги кода
\usepackage{listings}
\lstset{
    basicstyle=\ttfamily\footnotesize,
    breaklines=true,
    frame=single,
    numbers=left,
    numberstyle=\tiny,
    tabsize=4,
    captionpos=b
}
\renewcommand{\lstlistingname}{Листинг}

% Теоремы и определения
\newtheorem{theorem}{Теорема}[chapter]
\newtheorem{lemma}[theorem]{Лемма}
\newtheorem{proposition}[theorem]{Утверждение}
\newtheorem{corollary}[theorem]{Следствие}

\theoremstyle{definition}
\newtheorem{definition}[theorem]{Определение}
\newtheorem{example}[theorem]{Пример}

\theoremstyle{remark}
\newtheorem{remark}[theorem]{Замечание}
\newtheorem{note}[theorem]{Примечание}

% Правильные кавычки
\usepackage{csquotes}

% Переносы
\usepackage{microtype}

% Приложения
\usepackage[title,titletoc]{appendix}

% Счетчики для формул по главам
\numberwithin{equation}{chapter}

% Переопределение названий
\addto\captionsrussian{%
    \renewcommand{\figurename}{Рисунок}%
    \renewcommand{\tablename}{Таблица}%
    \renewcommand{\contentsname}{Содержание}%
    \renewcommand{\bibname}{Список литературы}%
    \renewcommand{\chaptername}{Глава}%
    \renewcommand{\appendixname}{Приложение}%
}
