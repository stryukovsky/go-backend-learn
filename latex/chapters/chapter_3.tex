\newpage
\begin{center}
	\textbf{\large 3. ВЛИЯНИЕ ВЫПУКЛОСТИ ТОРГОВОГО МНОЖЕСТВА НА ПОВЕДЕНИЕ ПОСТАВЩИКОВ ЛИКВИДНОСТИ}
\end{center}
\refstepcounter{chapter}
\addcontentsline{toc}{chapter}{3. ВЛИЯНИЕ ВЫПУКЛОСТИ ТОРГОВОГО МНОЖЕСТВА НА ПОВЕДЕНИЕ ПОСТАВЩИКОВ ЛИКВИДНОСТИ}

\section{Задача арбитража}

Задача арбитража неизбежно возникает при рассмотрении любой торговой платформы
или биржи. В общем виде, задача формулируется как максимизация выхода от
сделки, вычисленного с помощью внешнего источника цен активов. Арбитраж по
определению невозможен без внешнего (относительно рассматриваемой платформы)
источника данных о ценах на активы.

\begin{equation}
	\max{\mathbf{c} \cdot (\lambda - \delta)}
	\label{eq:arbitrage_problem_general}
\end{equation}

Где $(\delta; \lambda) \in T(\mathbf{R})$ -- валидная сделка из торгового
множества для текущего состояния пула $\mathbf{R}$.  Можно иначе выразить,
через $\mathbf{R}$ и $\mathbf{R'}$, где $\mathbf{R'}$ -- состояние пула после
одной сделки $(\delta; \lambda) \in T(R)$:

\begin{equation}
	\max{\mathbf{c} \cdot (\mathbf{R} - \mathbf{R'})}
	\label{eq:arbitrage_problem_general_via_states}
\end{equation}

Почему такое выражение возможно? $\mathbf{R'} = \mathbf{R} + \delta - \lambda$,
т.к. когда сделка проходит, из пула изымается $\lambda$ токенов и добавляется
$\delta$ токенов.  Если выразить $\delta - \lambda = \mathbf{R'} - \mathbf{R}$,
а затем поменять знаки, получим, что $\lambda - \delta = \mathbf{R} -
	\mathbf{R'}$, т.е. вместо $\lambda - \delta$ можно подставить $\mathbf{R} -
	\mathbf{R'}$ и наоборот.

Функция $f(R') = \mathbf{c} \cdot (\mathbf{R} - \mathbf{R'})$ является линейной
по $R'$, максимальное значение функции $f$ будет в точке минимального значения
$\mathbf{c} \cdot \mathbf{R'}$.

\section{Множество оптимальных решений выпукло}

Утверждение: Если максимизируется значение вогнутой функцию (или минимизируем
выпуклую) на выпуклом множестве, то локальный оптимум является глобальным
оптимумом.

Доказательство единственности и выпуклости множества решений:
Пусть $\mathbf{R_{1}^{*}}$ и $\mathbf{R_{2}^{*}}$ — два оптимума для которых
$\mathbf{c} \cdot \mathbf{R_{1}^{*}} = \mathbf{c} \cdot \mathbf{R_{2}^{*}} =
	v$.

Для любого $\theta \in [0,1]$, пусть $\mathbf{R_{\theta}} = \theta \cdot
	\mathbf{R_{1}^{*}} + (1 - \theta) \cdot \mathbf{R_{2}^{*}}$. Тогда

$\mathbf{c} \cdot \mathbf{R_{\theta}}  = \mathbf{c} \cdot (\theta \cdot
	\mathbf{R_{1}^{*}} + (1 - \theta) \cdot \mathbf{R_{2}^{*}}) = \theta \cdot
	\mathbf{c} \cdot \mathbf{R_{1}^{*}} + (1 - \theta) \cdot \mathbf{c} \cdot
	\mathbf{R_{2}^{*}} = \theta \cdot v + (1 - \theta) \cdot v = v$

Тогда $\mathbf{R_{\theta}}$ -- тоже оптимум.

\section{Задача арбитража в UniswapV3}
Для UniswapV3 можно сформулировать задачу арбитража следующим образом:

\begin{equation}
	\max{(c_2 \cdot \lambda_2 - c_1 \cdot \delta_1)}
	\label{eq:arbitrage_problem_uniswapv3}
\end{equation}

Где $c_1$ -- внешняя цена актива 1 ($\mathbf{R_1}$ в <<терминах>> резервов), а
$c_2$ -- цена актива 2 соответственно. Внешняя цена поставляется, например,
сторонней торговой площадкой, централизованной биржей и т.д.  Например, если в
пуле ликвидности находятся активы Wrapped Ethereum и Wrapped Bitcoin, то
внешние цены таких активов удобно представлять в виде цены в долларах США или
рублях за единицу актива.

Идея формулизации проста: арбитражник интересуется заработком на разнице в
курсе активов на разных площадках. В рассматриваемой постановке задачи
арбитражник покупает актив 2, а продает актив 1, хотя для UniswapV3 это
непринципиально, и можно сформулировать наоборот. Доход арбитражника максимален
тогда, когда максимальна разница между тем, за сколько он купил актив 2 и за
сколько он продал актив 1.

При помощи \eqref{eq:trade_fn_uniswapv3}, можно выразить задачу арбитража из
торговой функции UniswapV3:

\begin{equation}
	\lambda_2 = R_2 - \frac{L^2}{R_1 + \delta_1}
	\label{eq:lambda_2_from_uniswapv3_trading_fn}
\end{equation}

Функция прибыли арбитражника для произвольного обменника с двумя токенами
выглядит следующим образом:

\begin{equation}
	\pi(\delta_1) = c_2 \cdot \lambda_2 - c_1 \cdot \delta_1
	\label{eq:general_uniswapv3_arbitrage_income}
\end{equation}

Чтобы получить функцию от одного аргумента, нужно подставить
\eqref{eq:lambda_2_from_uniswapv3_trading_fn} в
\eqref{eq:general_uniswapv3_arbitrage_income}:

\begin{equation}
	\pi(\delta_1) = c_2 \cdot (R_2 - \frac{L^2}{R_1 + \delta_1}) - c_1 \cdot \delta_1
	\label{eq:uniswapv3_arbitrage_income}
\end{equation}

Арбитражника можно рассматривать как рационального <<игрока>>, который
стремится максимизировать свою прибыль. Для этого нужно найти производную:

\begin{equation}
	\frac{d\pi}{d\delta_1} = c_2 \cdot \frac{L^2}{(R_1 + \delta_1)^2} - c_1
	\label{eq:derivative_uniswapv3_arbitrage_income}
\end{equation}

И затем найти точку оптимума, $\frac{d\pi}{d\delta_1} = 0$:

\begin{gather}
	c_2 \cdot \frac{L^2}{(R_1 + \delta_1)^2} - c_1 = 0 \\
	c_2 \cdot \frac{L^2}{(R_1 + \delta_1)^2} = c_1 \\
	(R_1 + \delta_1)^2 = \frac{L^2 \cdot c_2}{c_1} \\
	R_1 + \delta_1 = L \cdot \sqrt{\frac{c_2}{c_1}} \\
	\delta_1^* = L \cdot \sqrt{\frac{c_2}{c_1}} - R_1
	\label{eq:maximum_arbitrage_income_uniswapv3}
\end{gather}

Таким образом, арбитражник в случае возникновения ситуации арбитража (задачи
арбитража) может предоставить $\delta_1^*$ токенов, и взамен получит
максимально возможный доход в данной ситуации.

\section{Влияние арбитража на стратегии предоставления ликвидности}

\subsection{Выпуклость торгового множества и возникновение арбитражных возможностей}

Задача арбитража сводится к максимизации линейной функции прибыли над выпуклым
торговым множеством $T(\mathbf{R})$. Выпуклость множества достижимых состояний
резервов является ключевым свойством CFMM и гарантирует существование
единственного (или выпуклого множества) оптимального решения $\delta_1^*$, при
котором цена в пуле выравнивается с внешней рыночной ценой $c = (c_1, c_2)$.

Как уже показано выше, в произвольный момент времени и произвольное состояние
пула всегда можно вычислить такое $\delta_1^*$. Таким образом, выпуклость
торгового множества не только допускает, но и структурирует арбитраж,
обеспечивая его предсказуемость и эффективность.

Хотя арбитражные операции извлекают ценность из расхождения цен, они
одновременно генерируют значительный объём торговли в пуле. Поскольку
поставщики ликвидности получают комиссионное вознаграждение пропорционально
своей доле в активной ликвидности, арбитраж становится косвенным, но
систематическим источником дохода для LP.

\subsection{Стратегическая адаптация институциональных LP под арбитраж}

Институциональные участники могут использовать следующие стратегии для
максимизации дохода:

\begin{enumerate}
	\item Концентрация ликвидности вблизи текущей цены. Поскольку арбитраж
	      всегда стремится вернуть внутреннюю цену к внешней, он происходит
	      преимущественно в окрестности рыночной цены. Узкий диапазон $[p_a, p_b]$ вокруг
	      $p$ позволяет LP захватывать максимальную долю арбитражного потока.

	\item Динамическое ребалансирование. Институционалы используют алгоритмы для
	      отслеживания движения цены и автоматического перемещения позиций, оставаясь «в
	      зоне действия» арбитража.

	\item Синергия с MEV-инфраструктурой. Владение или доступ к mempool-аналитике
	      позволяет предугадывать направление арбитражных свопов и размещать ликвидность
	      заранее.
\end{enumerate}

Таким образом, арбитраж не только корректирует цены, но и формирует стимулы для
специфического поведения LP: чем точнее позиция отслеживает рыночную цену, тем
выше её доля в комиссиях. Это создаёт асимметрию между участниками:
институционалы, обладающие технологическими и информационными преимуществами,
систематически получают более высокую доходность, чем розничные LP,
использующие статичные стратегии.

Следовательно, арбитраж в Uniswap V3 является не просто механизмом
эффективности рынка, но и фактором усиления неравенства среди поставщиков
ликвидности.


\subsection{Техническая особенность арбитража} Важным остается вопрос о том,
как можно определить, что та или иная сделка является частью арбитражных
действий. В рамках данной работы это потребуется, чтобы для массива сделок в
произвольном пула Uniswap можно было эффективно подсчитывать процент сделок
арбитража относительно всех сделок в пуле. Без технической возможности
определять процент арбитражных сделок в пуле предыдущее утверждение о влиянии
арбитража на поведение институциональных участников рынка нельзя ни
подтвердить, ни опровергнуть.

Основной идеей определения того, что сделка является арбитражной, является тот
факт, что покупка первого токена в количестве $r_1$ (и продажа второго токена в
количестве $r_2$)  происходит в одном блоке с обратным выкупом второго токена
$r_2'$ (и продажей только что купленного первого токена $r_1$). $r_2' < r_2$
из-за комиссии и потерь проскальзывания. Формально это условие выглядит так:
сделка $forward = (\delta, \lambda) = (r_2, r_1)$ и следующая за ней сделка
выкупа $buyback = (\delta, \lambda) = (r_1, r_2')$ происходят в одном блоке или
в соседствующих блоках.

Для формализации критерия <<соседства>> двух арбитражных операций введём
понятие окрестности блоков. Пусть $b$ — некоторый блок с номером $n_b \in
	\mathbb{N}$. Тогда окрестностью радиуса $s \in \mathbb{N}$ вокруг блока $b$
называется множество блоков
\[
	O_s(b) = \left\{ b' \in \mathcal{B} \;\middle|\; |n_{b'} - n_b| < s \right\},
\]
где $\mathcal{B}$ — множество всех блоков в цепочке, а $n_{b'}$ — номер блока $b'$.

\subsection{Выбор размера окрестности блока}

В пулах с низкой волатильностью (например, $USDC/USDT$) цена колеблется в узком
диапазоне, и арбитражные возможности сохраняются в течение нескольких блоков.
Это обусловлено тем, что расхождение между внутренней ценой пула и внешним
рынком (например, CEX) не устраняется мгновенно, а требует серии микросвопов. В
таких условиях арбитражники могут позволить себе задержку в 1–5 блоков между
входом и выходом из позиции, не теряя прибыльности.

Напротив, в пулах с высокой волатильностью (например, $ETH/USDC$) цена
подвержена резким изменениям, и арбитражное окно остаётся открытым лишь в
течение одного или даже части одного блока. Любая задержка приводит к тому, что
цена смещается, и первоначальная сделка перестаёт быть прибыльной.
Следовательно, для volatile-пулов адекватным выбором будет $s = 1$, тогда как
для стейблкоиновых пулов допустимо $s \in [2, 5]$.

