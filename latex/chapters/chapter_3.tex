\newpage
\begin{center}
	\textbf{\large 3. ВЛИЯНИЕ ВЫПУКЛОСТИ ТОРГОВОГО МНОЖЕСТВА НА ПОВЕДЕНИЕ ПОСТАВЩИКОВ ЛИКВИДНОСТИ}
\end{center}
\refstepcounter{chapter}
\addcontentsline{toc}{chapter}{3. ВЛИЯНИЕ ВЫПУКЛОСТИ ТОРГОВОГО МНОЖЕСТВА НА ПОВЕДЕНИЕ ПОСТАВЩИКОВ ЛИКВИДНОСТИ}

\section{Задача арбитража}

Задача арбитража неизбежно возникает при рассмотрении любой торговой платформы или биржи. В общем виде, задача формулируется как максимизация выхода от сделки, вычисленного с помощью внешнего источника цен активов. Арбитраж по определению невозможен без внешнего (относительно рассматриваемой платформы) источника данных о ценах на активы.

\begin{equation}
	\max{\mathbf{c} \cdot (\lambda - \delta)}
	\label{eq:arbitrage_problem_general}
\end{equation}

Где $(\delta; \lambda) \in T(\mathbf{R})$ -- валидная сделка из торгового множества для текущего состояния пула $\mathbf{R}$.  
Можно иначе выразить, через $\mathbf{R}$ и $\mathbf{R'}$, где $\mathbf{R'}$ -- состояние пула после одной сделки $(\delta; \lambda) \in T(R)$: 

\begin{equation}
\max{\mathbf{c} \cdot (\mathbf{R} - \mathbf{R'})}
	\label{eq:arbitrage_problem_general_via_states}
\end{equation}

Почему такое выражение возможно? $\mathbf{R'} = \mathbf{R} + \delta - \lambda$, т.к. когда сделка проходит, из пула изымается $\lambda$ токенов и добавляется $\delta$ токенов.  
Если выразить $\delta - \lambda = \mathbf{R'} - \mathbf{R}$, а затем поменять знаки, получим, что $\lambda - \delta = \mathbf{R} - \mathbf{R'}$, т.е. вместо $\lambda - \delta$ можно подставить $\mathbf{R} - \mathbf{R'}$ и наоборот. 

Функция $f(R') = \mathbf{c} \cdot (\mathbf{R} - \mathbf{R'})$ является линейной по $R'$, максимальное значение функции $f$ будет в точке минимального значения $\mathbf{c} \cdot \mathbf{R'}$.  

Теорема о выпуклой оптимизации

\section{Множество оптимальных решений выпукло}

Утверждение: Если максимизируется значение вогнутой функцию (или минимизируем выпуклую) на выпуклом множестве, то локальный оптимум является глобальным оптимумом.

Доказательство единственности и выпуклости множества решений:
Пусть $\mathbf{R_{1}^{*}}$ и $\mathbf{R_{2}^{*}}$ — два оптимума для которых $\mathbf{c} \cdot \mathbf{R_{1}^{*}} = \mathbf{c} \cdot \mathbf{R_{2}^{*}} = v$.  

Для любого $\theta \in [0,1]$, пусть $\mathbf{R_{\theta}} = \theta \cdot \mathbf{R_{1}^{*}} + (1 - \theta) \cdot \mathbf{R_{2}^{*}}$. Тогда 

$\mathbf{c} \cdot \mathbf{R_{\theta}}  = \mathbf{c} \cdot (\theta \cdot \mathbf{R_{1}^{*}} + (1 - \theta) \cdot \mathbf{R_{2}^{*}}) = \theta \cdot \mathbf{c} \cdot \mathbf{R_{1}^{*}} + (1 - \theta) \cdot \mathbf{c} \cdot \mathbf{R_{2}^{*}} = \theta \cdot v + (1 - \theta) \cdot v = v$

Тогда $\mathbf{R_{\theta}}$ -- тоже оптимум.  

\section{Задача арбитража в UniswapV3}
Для UniswapV3 можно сформулировать задачу арбитража следующим образом:

\begin{equation}
\max{(c_2 \cdot \lambda_2 - c_1 \cdot \delta_1)}
	\label{eq:arbitrage_problem_uniswapv3}
\end{equation}

Где $c_1$ -- внешняя цена актива 1 ($\mathbf{R_1}$ в <<терминах>> резервов), а $c_2$ -- цена актива 2 соответственно. Внешняя цена поставляется, например, сторонней торговой площадкой, централизованной биржей и т.д.  Например, если в пуле ликвидности находятся активы Wrapped Ethereum и Wrapped Bitcoin, то внешние цены таких активов удобно представлять в виде цены в долларах США или рублях за единицу актива.  

Идея формулизации проста: арбитражник интересуется заработком на разнице в курсе активов на разных площадках. В рассматриваемой постановке задачи арбитражник покупает актив 2, а продает актив 1, хотя для UniswapV3 это непринципиально, и можно сформулировать наоборот. Доход арбитражника максимален тогда, когда максимальна разница между тем, за сколько он купил актив 2 и за сколько он продал актив 1.

При помощи \eqref{eq:trade_fn_uniswapv3}, можно выразить задачу арбитража из торговой функции UniswapV3:

\begin{equation}
  \lambda_2 = R_2 - \frac{L^2}{R_1 + \delta_1}
	\label{eq:lambda_2_from_uniswapv3_trading_fn}
\end{equation}

Функция прибыли арбитражника для произвольного обменника с двумя токенами выглядит следующим образом:

\begin{equation}
  \pi(\delta_1) = c_2 \cdot \lambda_2 - c_1 \cdot \delta_1
	\label{eq:general_uniswapv3_arbitrage_income}
\end{equation}

Чтобы получить функцию от одного аргумента, нужно подставить \eqref{eq:lambda_2_from_uniswapv3_trading_fn} в \eqref{eq:general_uniswapv3_arbitrage_income}:

\begin{equation}
\pi(\delta_1) = c_2 \cdot (R_2 - \frac{L^2}{R_1 + \delta_1}) - c_1 \cdot \delta_1
	\label{eq:uniswapv3_arbitrage_income}
\end{equation}

Арбитражника можно рассматривать как рационального <<игрока>>, который стремится максимизировать свою прибыль. Для этого нужно найти производную:

\begin{equation}
\frac{d\pi}{d\delta_1} = c_2 \cdot \frac{L^2}{(R_1 + \delta_1)^2} - c_1
	\label{eq:derivative_uniswapv3_arbitrage_income}
\end{equation}

И затем найти точку оптимума, $\frac{d\pi}{d\delta_1} = 0$:

\begin{gather}
c_2 \cdot \frac{L^2}{(R_1 + \delta_1)^2} - c_1 = 0 \\
c_2 \cdot \frac{L^2}{(R_1 + \delta_1)^2} = c_1 \\
(R_1 + \delta_1)^2 = \frac{L^2 \cdot c_2}{c_1} \\
R_1 + \delta_1 = L \cdot \sqrt{\frac{c_2}{c_1}} \\
\delta_1^* = L \cdot \sqrt{\frac{c_2}{c_1}} - R_1 
	\label{eq:maximum_arbitrage_income_uniswapv3}
\end{gather}

Таким образом, арбитражник в случае возникновения ситуации арбитража (задачи арбитража) может предоставить $\delta_1^*$ токенов, и взамен получит максимально возможный доход в данной ситуации. 


