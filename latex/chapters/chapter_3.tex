\newpage
\begin{center}
	\textbf{\large 3. ВЛИЯНИЕ ВЫПУКЛОСТИ ТОРГОВОГО МНОЖЕСТВА НА ПОВЕДЕНИЕ ПОСТАВЩИКОВ ЛИКВИДНОСТИ}
\end{center}
\refstepcounter{chapter}
\addcontentsline{toc}{chapter}{3. ВЛИЯНИЕ ВЫПУКЛОСТИ ТОРГОВОГО МНОЖЕСТВА НА ПОВЕДЕНИЕ ПОСТАВЩИКОВ ЛИКВИДНОСТИ}

\section{Задача арбитража}

Задача арбитража неизбежно возникает при рассмотрении любой торговой платформы или биржи. В общем виде, задача формулируется как максимизация выхода от сделки, вычисленного с помощью внешнего источника цен активов. Арбитраж по определению невозможен без внешнего (относительно рассматриваемой платформы) источника данных о ценах на активы.

\begin{equation}
	\max{\mathbf{c} \cdot (\lambda - \delta)}
	\label{eq:arbitrage_problem_general}
\end{equation}

Где $(\delta; \lambda) \in T(\mathbf{R})$ -- валидная сделка из торгового множества для текущего состояния пула $\mathbf{R}$

Для UniswapV3 можно сформулировать задачу арбитража следующим образом:

\begin{equation}
\max{(c_2 \cdot \lambda_2 - c_1 \cdot \delta_1)}
	\label{eq:arbitrage_problem_uniswap_v3}
\end{equation}

Где $c_1$ -- внешняя цена актива 1 ($\mathbf{R_1}$ в <<терминах>> резервов), а $c_2$ -- цена актива 2 соответственно. Внешняя цена поставляется, например, сторонней торговой площадкой, централизованной биржей и т.д.  Например, если в пуле ликвидности находятся активы Wrapped Ethereum и Wrapped Bitcoin, то внешние цены таких активов удобно представлять в виде цены в долларах США или рублях за единицу актива.  

Идея формулизации проста: арбитражник интересуется заработком на разнице в курсе активов на разных площадках. В рассматриваемой постановке арбитражник покупает актив 2, а продает актив 1, хотя для UniswapV3 это непринципиально, и можно сформулировать наоборот. Доход арбитражника максимален тогда, когда максимальна разница между тем, за сколько он купил актив 2 и за сколько он продал актив 1.



