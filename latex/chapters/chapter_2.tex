\newpage
\begin{center}
  \textbf{\large 2. Предоставление ликвидности в Uniswap V3}
\end{center}
\refstepcounter{chapter}
\addcontentsline{toc}{chapter}{2. ПРЕДОСТАВЛЕНИЕ ЛИКВИДНОСТИ В UNISWAP V3}

\section{Концентрированная ликвидность}

Ключевым отличием Uniswap V3 от предыдущих версий является введение концентрированной ликвидности (concentrated liquidity). В отличие от Uniswap V2, где ликвидность распределялась равномерно по всему ценовому диапазону $(0, \infty)$, Uniswap V3 позволяет поставщикам ликвидности концентрировать свой капитал в определенных ценовых интервалах.

Математически это означает, что функция торговли $\varphi$ из уравнения \eqref{eq3} применяется не ко всему возможному ценовому диапазону, а только к выбранному интервалу $[P_a, P_b]$, где $P_a$ и $P_b$ -- нижняя и верхняя границы ценового диапазона соответственно.

Для позиции с концентрированной ликвидностью в диапазоне $[P_a, P_b]$ функция торговли принимает вид:

\begin{equation}
\varphi(R, \delta, \lambda) = 
\begin{cases}
(R_1 + \delta_1 - \lambda_1) \cdot (R_2 + \delta_2 - \lambda_2) & \text{если } P_a \leq P \leq P_b \\
0 & \text{иначе}
\end{cases}
\label{eq:concentrated_trading}
\end{equation}

где $P = \frac{R_2}{R_1}$ -- текущая цена в пуле.

\section{Тики и дискретизация ценового пространства}

Для реализации концентрированной ликвидности непрерывное ценовое пространство разбивается на дискретные интервалы, называемые тиками (ticks). Тики представляют собой границы между дискретными областями в ценовом пространстве.

Цена для $i$-го тика определяется как:

\begin{equation}
P(i) = 1.0001^i
\label{eq:tick_price}
\end{equation}

где $i \in \mathbb{Z}$ -- индекс тика в диапазоне $[-887272, 887272]$.

Изменение цены на один тик соответствует изменению на $0.01\%$ или один базисный пункт (basis point):

\begin{equation}
\frac{P(i+1) - P(i)}{P(i)} = \frac{1.0001^{i+1} - 1.0001^i}{1.0001^i} = 0.0001 = 0.01\%
\label{eq:tick_change}
\end{equation}

\subsection{Интервалы тиков}

Не каждый тик может быть инициализирован. Вместо этого каждый пул инициализируется с интервалом тиков (tick spacing), который определяет расстояние между тиками. Интервал тиков зависит от уровня комиссии пула:

\begin{table}[h]
\centering
\begin{tabular}{|c|c|c|}
\hline
\textbf{Комиссия} & \textbf{Интервал тиков} & \textbf{Минимальное изменение цены} \\
\hline
0.01\% & 1 & 0.01\% \\
0.05\% & 10 & 0.10\% \\
0.30\% & 60 & 0.60\% \\
1.00\% & 200 & 2.00\% \\
\hline
\end{tabular}
\caption{Соответствие между уровнями комиссии и интервалами тиков}
\label{tab:tick_spacing}
\end{table}

Для данного интервала тиков $ts$ только тики с индексами, кратными $ts$, могут быть инициализированы:

\begin{equation}
i \equiv 0 \pmod{ts}
\label{eq:tick_divisibility}
\end{equation}

\section{Позиция поставщика ликвидности}

Позиция поставщика ликвидности в Uniswap V3 определяется тремя параметрами:

\begin{enumerate}
\item Количество ликвидности $L \in \mathbb{R}_+$
\item Нижняя граница $i_a \in \mathbb{Z}$ (нижний тик)
\item Верхняя граница $i_b \in \mathbb{Z}$ (верхний тик)
\end{enumerate}

где $i_a$ и $i_b$ должны быть кратны интервалу тиков $ts$: $i_a \equiv i_b \equiv 0 \pmod{ts}$.

\subsection{Виртуальные и реальные резервы}

Для позиции с концентрированной ликвидностью вводятся понятия виртуальных и реальных резервов. Пусть:

\begin{itemize}
\item $x_r, y_r$ -- реальные резервы токенов в позиции
\item $x_v, y_v$ -- виртуальные резервы, соответствующие полному диапазону $(0, \infty)$
\item $P = P(i)$ -- текущая цена
\item $P_a = P(i_a)$ -- цена нижней границы
\item $P_b = P(i_b)$ -- цена верхней границы
\end{itemize}

Связь между реальными и виртуальными резервами выражается через формулы:

\begin{equation}
L = \sqrt{x_v \cdot y_v} = \sqrt{x_r \cdot y_r} \cdot \frac{1}{\sqrt{P} - \sqrt{P_a}} \cdot \frac{1}{\frac{1}{\sqrt{P}} - \frac{1}{\sqrt{P_b}}}
\label{eq:liquidity_virtual}
\end{equation}

При текущей цене $P \in [P_a, P_b]$ количества токенов в позиции определяются как:

\begin{equation}
x_r = L \cdot \left(\frac{1}{\sqrt{P}} - \frac{1}{\sqrt{P_b}}\right)
\label{eq:real_x}
\end{equation}

\begin{equation}
y_r = L \cdot \left(\sqrt{P} - \sqrt{P_a}\right)
\label{eq:real_y}
\end{equation}

\section{Расчет количества ликвидности}

Рассмотрим задачу определения количества ликвидности $L$ для позиции, когда поставщик ликвидности предоставляет определенные количества токенов $\Delta x$ и $\Delta y$.

\subsection{Случай 1: Текущая цена внутри диапазона}

Если текущая цена $P \in [P_a, P_b]$, то ликвидность рассчитывается из двух условий:

\begin{equation}
L_x = \Delta x \cdot \frac{\sqrt{P} \cdot \sqrt{P_b}}{\sqrt{P_b} - \sqrt{P}}
\label{eq:liquidity_from_x}
\end{equation}

\begin{equation}
L_y = \frac{\Delta y}{\sqrt{P} - \sqrt{P_a}}
\label{eq:liquidity_from_y}
\end{equation}

Итоговое количество ликвидности определяется как минимум:

\begin{equation}
L = \min(L_x, L_y)
\label{eq:liquidity_min}
\end{equation}

\subsection{Случай 2: Текущая цена ниже диапазона}

Если $P < P_a$, то позиция состоит только из токена X:

\begin{equation}
L = \Delta x \cdot \frac{\sqrt{P_a} \cdot \sqrt{P_b}}{\sqrt{P_b} - \sqrt{P_a}}
\label{eq:liquidity_below}
\end{equation}

\subsection{Случай 3: Текущая цена выше диапазона}

Если $P > P_b$, то позиция состоит только из токена Y:

\begin{equation}
L = \frac{\Delta y}{\sqrt{P_b} - \sqrt{P_a}}
\label{eq:liquidity_above}
\end{equation}

\section{Пример расчета позиции}

Рассмотрим конкретный пример. Пусть поставщик ликвидности хочет создать позицию в пуле ETH/USDC со следующими параметрами:

\begin{itemize}
\item Текущая цена: $P = 2000$ USDC/ETH
\item Нижняя граница: $P_a = 1500$ USDC/ETH
\item Верхняя граница: $P_b = 2500$ USDC/ETH
\item Предоставляемые средства: 1 ETH и 2000 USDC
\end{itemize}

\subsection{Шаг 1: Вычисление квадратных корней цен}

\begin{align}
\sqrt{P} &= \sqrt{2000} \approx 44.721 \\
\sqrt{P_a} &= \sqrt{1500} \approx 38.730 \\
\sqrt{P_b} &= \sqrt{2500} = 50.000
\end{align}

\subsection{Шаг 2: Расчет ликвидности}

Из уравнений \eqref{eq:liquidity_from_x} и \eqref{eq:liquidity_from_y}:

\begin{align}
L_x &= 1 \cdot \frac{44.721 \cdot 50.000}{50.000 - 44.721} \approx 423.61 \\
L_y &= \frac{2000}{44.721 - 38.730} \approx 333.89
\end{align}

Следовательно, $L = \min(423.61, 333.89) = 333.89$.

\subsection{Шаг 3: Определение фактически используемых количеств}

При $L = 333.89$ фактически используемые количества токенов:

\begin{align}
x_{used} &= 333.89 \cdot \frac{50.000 - 44.721}{44.721 \cdot 50.000} \approx 0.787 \text{ ETH} \\
y_{used} &= 333.89 \cdot (44.721 - 38.730) = 2000 \text{ USDC}
\end{align}

Таким образом, будет использовано 0.787 ETH и 2000 USDC, а оставшиеся 0.213 ETH не будут включены в позицию.

\section{NFT-представление позиций}

В Uniswap V3 каждая позиция ликвидности представлена как невзаимозаменяемый токен (NFT) стандарта ERC-721. Это принципиальное отличие от Uniswap V2, где позиции представлялись взаимозаменяемыми токенами ERC-20.

NFT-представление позиций предоставляет следующие преимущества:

\begin{enumerate}
\item \textbf{Уникальность}: Каждая позиция уникальна по своим параметрам $(L, i_a, i_b)$
\item \textbf{Композируемость}: Позиции могут быть интегрированы в другие DeFi-протоколы
\item \textbf{Передаваемость}: Позиции могут быть переданы или проданы как самостоятельные активы
\end{enumerate}

Контракт \texttt{NonfungiblePositionManager} управляет созданием, модификацией и уничтожением позиций.

\section{Активная и неактивная ликвидность}

Важным понятием в Uniswap V3 является различие между активной и неактивной ликвидностью.

\subsection{Активная ликвидность}

Ликвидность позиции считается активной, если текущая цена $P$ находится в диапазоне позиции $[P_a, P_b]$:

\begin{equation}
\text{Позиция активна} \Leftrightarrow P \in [P_a, P_b]
\label{eq:active_position}
\end{equation}

Только активные позиции зарабатывают торговые комиссии.

\subsection{Изменение состава позиции}

Когда цена движется через диапазон позиции, состав активов в позиции изменяется. При движении цены вверх (токен Y дорожает относительно токена X):

\begin{itemize}
\item Количество токена X уменьшается
\item Количество токена Y увеличивается
\item При $P \geq P_b$ позиция полностью состоит из токена Y
\end{itemize}

Формально, при изменении цены от $P$ до $P'$ изменения в количествах токенов составляют:

\begin{equation}
\Delta x = L \cdot \left(\frac{1}{\sqrt{P'}} - \frac{1}{\sqrt{P}}\right)
\label{eq:delta_x}
\end{equation}

\begin{equation}
\Delta y = L \cdot \left(\sqrt{P'} - \sqrt{P}\right)
\label{eq:delta_y}
\end{equation}

\section{Эффективность капитала}

Концентрированная ликвидность обеспечивает значительное повышение эффективности использования капитала. В зависимости от уровня концентрации, эффективность капитала может увеличиться до 4000 раз по сравнению с Uniswap V2.

Коэффициент эффективности капитала для позиции в диапазоне $[P_a, P_b]$ по сравнению с полным диапазоном $(0, \infty)$ можно выразить как:

\begin{equation}
\text{Эффективность} = \frac{L_{full}}{L_{concentrated}} = \frac{\sqrt{P_b} - \sqrt{P_a}}{\sqrt{P_b \cdot P_a}} \cdot \sqrt{P}
\label{eq:capital_efficiency}
\end{equation}

где $L_{full}$ -- количество ликвидности, необходимое для обеспечения той же глубины рынка на полном диапазоне.

\section{Стратегии предоставления ликвидности}

\subsection{Узкий диапазон}

Стратегия узкого диапазона предполагает концентрацию ликвидности в небольшом интервале вокруг текущей цены:

\begin{equation}
\frac{P_b - P_a}{P} \ll 1
\label{eq:narrow_range}
\end{equation}

Преимущества:
\begin{itemize}
\item Максимальная эффективность капитала
\item Высокий доход от комиссий при активной торговле
\end{itemize}

Недостатки:
\begin{itemize}
\item Высокий риск выхода цены из диапазона
\item Необходимость частой ребалансировки
\item Повышенный непостоянный убыток
\end{itemize}

\subsection{Широкий диапазон}

Стратегия широкого диапазона предполагает охват значительной части ценового пространства:

\begin{equation}
\frac{P_b - P_a}{P} \gg 1
\label{eq:wide_range}
\end{equation}

Преимущества:
\begin{itemize}
\item Меньший риск выхода цены из диапазона
\item Меньшая необходимость в управлении позицией
\item Пониженный непостоянный убыток
\end{itemize}

Недостатки:
\begin{itemize}
\item Меньшая эффективность капитала
\item Меньший доход от комиссий на единицу капитала
\end{itemize}

\subsection{Стратегия для стейблкоинов}

Для пар стейблкоинов, таких как USDC/USDT, оптимальной является стратегия очень узкого диапазона:

\begin{equation}
P_a = 0.99, \quad P_b = 1.01
\label{eq:stablecoin_range}
\end{equation}

В паре DAI/USDC версии V2 только около 0.50\% доступного капитала использовалось для торговли в диапазоне от \$0.99 до \$1.01.

\section{Математическая модель дохода поставщика ликвидности}

Доход поставщика ликвидности в Uniswap V3 складывается из торговых комиссий и изменения стоимости активов в позиции.

\subsection{Накопление комиссий}

Комиссии накапливаются пропорционально доле ликвидности позиции в общей активной ликвидности и времени нахождения в активном состоянии:

\begin{equation}
\text{Комиссии} = \int_{t_0}^{t_1} \frac{L_{position}}{L_{active}(t)} \cdot \text{fee}(t) \cdot V(t) \, dt
\label{eq:fee_accumulation}
\end{equation}

где:
\begin{itemize}
\item $L_{position}$ -- ликвидность позиции
\item $L_{active}(t)$ -- общая активная ликвидность в момент времени $t$
\item $\text{fee}(t)$ -- уровень комиссии
\item $V(t)$ -- объем торгов в момент времени $t$
\end{itemize}

\subsection{Непостоянный убыток}

Концентрированные позиции могут испытывать более значительный непостоянный убыток, чем позиции V2, если цены выходят за пределы выбранного диапазона.

Для позиции в диапазоне $[P_a, P_b]$ при изменении цены от $P_0$ до $P_1$ (где $P_0, P_1 \in [P_a, P_b]$) непостоянный убыток составляет:

\begin{equation}
IL = \frac{2\sqrt{r}}{1+r} - 1
\label{eq:impermanent_loss}
\end{equation}

где $r = \frac{P_1}{P_0}$ -- отношение цен.

Если цена выходит за пределы диапазона, позиция полностью конвертируется в один из токенов, и непостоянный убыток может быть существенно выше.
