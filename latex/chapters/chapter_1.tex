\newpage
\begin{center}
  \textbf{\large 1. Маркетмейкер, использующий константное произведение }
\end{center}
\refstepcounter{chapter}
\addcontentsline{toc}{chapter}{1. МАРКЕТМЕЙКЕР, ИСПОЛЬЗУЮЩИЙ КОНСТАНТНОЕ ПРОИЗВЕДЕНИЕ}

\section{Использование маркетмейкеров в децентрализованных биржах}

Изначально децентрализованные биржи представляют собой метод обмена токенами между участниками блокчейн-протокола без необходимости доверять какому-либо централизованному источнику информации о ценах, объемах торгов и прочего. На текущий момент (конец 2025 года) порядка 200 миллиардов долларов участвовало в различных протоколах децентрализованного обмена. 
В общем виде, ключевым объектом децентрализованных бирж является маркетмейкер, основанный на константной функции (CFMM, constant function market makers). Успех Uniswap в свое время способствовал развитию популярности автоматизированных маркетмейкеров (AMM, automated market maker), основанных на функции константного произведения (constant product market maker). Маркет мейкер, основанный на константной функции, можно задать с помощью функции торговли $\varphi$:

\begin{equation}
\varphi: \mathbb{R}^n_{+} \times \mathbb{R}^n_{+} \times \mathbb{R}^n_{+} \rightarrow \mathbb{R}
\label{eq:trade_fn_general}
\end{equation}

Функция трех аргументов $\varphi$ формально задает процесс изменения состояния т.н. пула ликвидности (Liquidity Pool). Первый аргумент представляет собой вектор резервов $\mathbf{R} \subset \mathbb{R}^n_{+}$. Второй аргумент представляет собой вектор входов $\mathbf{\delta} \subset \mathbb{R}^n_{+}$, т.е. то, сколько токенов предоставил пользователь в пул ликвидности в процессе торговли. Третий аргумент представляет собой вектор выходов $\mathbf{\lambda} \subset \mathbb{R}^n_{+}$, т.е. то, сколько токенов получит пользователь от пула ликвидности в процессе торговли.
Возвращает функция количество ликвидности после совершения сделки.

Приведем конкретный пример. Пусть задан пул ликвидности Uniswap V3, в нем находятся токены WETH(обернутый Ethereum, ERC-20 адаптация нативного токена сети Ethereum Mainnet) и USDT (Tether USD, стейблкоин, чья стоимость примерно равна одному доллару США). В случае Uniswap V3, количество токенов в пуле $n = 2$

Эти токены находятся в пуле в определенном количестве, и это количество задает вектор резервов, например: $\mathbf{R} = \{1, 2000\}$. В пуле находятся 2000 USDT и 1 WETH.
Векторы $\delta$ и $\lambda$ задают т.н. сделку, которая состоит из вектора входов и вектора выходов. Например, если пользователь хочет купить WETH, предоставив 300 USDT, то эта сделка будет выглядеть так: предоставление 300 USDT $\delta = \{0, 300\}$ и получение взамен 0.1304 WETH $\lambda = \{0, 0.1304\}$

Почему именно 0.1304 WETH? Дело в том, что на функцию $\varphi$ накладывается важное ограничение, которое и называется постоянным произведения (constant product). Сделка $(\delta, \lambda)$ считается валидной, если

\begin{equation}
\varphi(R, \delta, \lambda) = \varphi(R, \mathbf{0}, \mathbf{0})  \label{eq:trade_fn_valid_criterion}
\end{equation}

Иными словами, сделка считается валидной, если общее количество ликвидности не меняется после совершения сделки. В нашем примере со сделкой покупки WETH за 300 USDT, количество WETH, которое получит пользователь, задается именно этим правилом.

Важное замечание заключается в том, что на практике это условие не соблюдается, и небольшие отклонения возникают вследствие несовершенства представления действительных чисел в любой вычислительной системе. Другой важной причиной нестрогого соблюдения это условия является так называемое проскальзывание, которое возникает в Uniswap V3, в протоколе, который рассматривается в первую очередь в данной работе.

\section{Задание функции торговли}
Чтобы определять, сколько пользователь получит токенов в результате совершения сделки, нужно задать функцию торговли.

Функция торговли для Uniswap V3 выглядит следующим образом

\begin{equation}
\varphi(R, \delta, \lambda) = (R_1 + \delta_1 - \lambda_1) \cdot (R_2 + \delta_2 - \lambda_2)
\label{eq:trade_fn_uniswapv3}
\end{equation}

где 
\begin{enumerate}
\item $R_1$ и $R_2$ представляют собой компоненты вектора резервов $\mathbf{R}$, количество токенов WETH и USDT в пуле соответственно;
\item $\delta_1$ и $\delta_2$ представляют собой количество WETH и USDT, которые пользователь предоставляет в пул в процессе сделки;
\item $\lambda_1$ и $\lambda_2$ представляют собой количество WETH и USDT, которые пользователь получит в результате сделки.
\end{enumerate}

В случае Uniswap V3 можно считать, что $\delta_1$ и $\delta_2$, равно как и $\lambda_1$ и $\lambda_2$ не могут быть одновременно ненулевыми.
Для целого ряда сделок вида "пользователь предоставляет USDT взамен на WETH" мы можем упростить функцию торговли:

\begin{equation}
\varphi(R, \delta, \lambda) = (R_1 - \lambda_1) \cdot (R_2 + \delta_2)
\label{eq:trade_fn_uniswapv3_simplified}
\end{equation}

В рамках таких сделок количество токена с индексом $1$ (WETH) уменьшается на $\lambda_1$, а количество токена с индексом $2$ (USDT) увеличивается на $\delta_2$.

\section{Пример сделки}
Стоит вернуться к примеру сделки, где пользователь покупает WETH за 300 USDT. 
До сделки количество ликвидности в пуле составляет:

\begin{equation}
\varphi(R, \mathbf{0}, \mathbf{0}) = (1 - 0) \cdot (2000 + 0) = 2000
\label{eq5}
\end{equation}

После сделки, согласно $\eqref{eq:trade_fn_valid_criterion}$, количество ликвидности должно остаться также $2000$.
Подставим в $\eqref{eq:trade_fn_uniswapv3_simplified}$ количество предоставляемых пользователем токенов USDT: $\delta_2 = 300$ и получим количество WETH $\lambda_1$, которые пользователь получит взамен:

\begin{equation}
\begin{gathered}
\varphi(R, \delta, \lambda) = (R_1 - \lambda_1) \cdot (R_2 + 300) = 2000
\\
(1 - \lambda_1) \cdot (2000 + 300) = 2000
\\
\lambda_1 = 1 - \frac{2000}{2300}
\\
\lambda_1 = 0.1304
\end{gathered}
\end{equation}

\section{Позиция поставщика ликвидности в Uniswap V3}
Данная работа детально рассматривает особенности работы протокола Uniswap третьей версии, поэтому стоит развить мысль о том, что такое позиция поставщика ликвидности (Liquidity provision position). Позиция поставщика ликвидности в Uniswap V3 определяется тремя параметрами:

\begin{enumerate}
\item Количество ликвидности $L \in \mathbb{R}_+$
\item Нижняя граница  цены $p_a$
\item Верхняя граница цены $p_b$
\end{enumerate}

Таким образом, позиция $P$ представляет собой структуру из $(L, p_a, p_b)$.

Позиция активна (также это свойство называется более близким к традиционным финансам In the money, англ. <<в деньгах>>) если текущая цена $p \in (p_a; p_b)$.

Когда позиция активна, поставщик ликвидности получает комиссию за предоставление ликвидности. Размер комиссии определяется самим пулом.
В Uniswap V3 размер комиссии в пуле зависит от <<природы>> токенов, которые в нем участвуют. 

Если оба токена являются стейблкоинами одной фиатной валюты (например, доллар США), то такой пул называют пулом стабильной пары (stable pool). В таком пуле комиссии, как правило, небольшие и составляют $0.01\%$, $0.05\%$. Комиссия списывается с каждой сделки обмена в том токене, который пользователь отдаёт при обмене в Uniswap V3.

\subsection{Виртуальные и реальные резервы}

Для позиций в UniswapV3 вводятся понятия виртуальных и реальных резервов. Пусть:

\begin{itemize}
\item $x_r, y_r$ -- реальные резервы токенов в позиции
\item $x_v, y_v$ -- виртуальные резервы, соответствующие полному диапазону $(0, \infty)$
\item $p$ -- текущая цена
\item $p_a$ -- цена нижней границы
\item $p_b$ -- цена верхней границы
\end{itemize}

Идею о виртуальных и реальных резервах проще представить в виде такого графика:

\begin{figure}[!h]
  \begin{center}
    \includegraphics[width=\textwidth]{Chapter1-virtual-and-real-reserves.pdf}
\caption{Представление идеи о виртуальных и реальных резервах. Две точки $a$ и $b$ символизируют на параболе точки, где цена составляет $p_a$ и $p_b$ соответственно. Точка $m$ представляет собой состояние пула ликвидности в настоящий момент времени. Исходя из того, что точка $m$ лежит <<между>> точками $a$ и $b$, можно сделать вывод о том, что эта позиция ликвидности является активной.}
    \label{figure:virtual_real_reserves}
  \end{center}
\end{figure}

На графике приведена ситуация, когда в пуле есть всего одна позиция провайдера ликвидности, и она в момент исполнения сделки является активной.
В данном примере на графике масштаб не приведен, поэтому пропорции не соблюдаются; в реальном случае если в пуле существует только одна позиция провайдера ликвидности, то $x_{real} = x_{virtual}$ равно как и $y_{real} = y_{virtual}$.

График схематично показывает геометрический смысл понятий виртуальной ликвидности и реальной ликвидности.
Виртуальная ликвидность показывает, сколько было бы токенов $x$ и $y$ доступно для обмена в состоянии точки $m$, если бы вся ликвидность пула была бы активной.

Реальная ликвидность показывает, сколько нужно <<потратить>> токенов $x$ или $y$ (в зависимости от того, какой токен пул <<отдает>> пользователю в рамках обмена токенов), чтобы <<выйти>> из позиции провайдера ликвидности. Можно представить, что точка текущего состояния пула $m$ <<идет>> вверх или вниз в зависимости от того, какой токен предоставляется пользователем и какой токен отдается пулом.

Требуется более формально задать виртуальную и реальную ликвидность. Проблема заключается в том, что в данной работе до сих пор не было представлено понятие о ликвидности как величине в протоколе UniswapV3. 

\subsection{Понятие ликвидности в UniswapV3}

В UniswapV3 для упрощения вычислений (в частности, виртуальная машина Ethereum (EVM) не поддерживает извлечение корня) де-факто используется не $xy$, а $\sqrt{xy}$. Но для упрощения повествования понятие ликвидности будет в себя включать и $xy$, и $\sqrt{xy}$. В местах, где эта разница имеет значение, ликвидность будет явно описана через формулу. Традиционно в работах и статьях по UniswapV3 $L=\sqrt{xy}$, и $L^2=\sqrt{xy}$.

Функция торговли $\varphi$, заданная в $\eqref{eq:trade_fn_uniswapv3_simplified}$, задает как раз значение $L^2$. Перефразируя, получим более "UniswapV3-like" выражение:

\begin{equation}
L^2 = (R_1 - \lambda_1) \cdot (R_2 + \delta_2)
\end{equation}

Если приравнять $\lambda_1 = 0$ и $\delta_2 = 0$, то получается функция исключительно от $R_1$ и $R_2$:


\begin{equation}
L^2 = (R_1) \cdot (R_2)
\label{liquidity_function_uniswapv3}
\end{equation}

В примере с графиком резервы назывались $x$ и $y$, разумеется, $x = R_1$ и $y = R_2$. Теперь возможно описать позицию провайдера ликвидности с точки зрения CPMM-терминов, что позволит более строго определить понятие виртуальной и реальной ликвидности.

Виртуальная ликвидность представляет собой вектор $(x_{virtual}, y_{virtual}$. Это резервы, которые используются для вычисления текущей цены активов в пуле. Именно эту цену видит конечный пользователь, который приходит на платформу UniswapV3 с целью обменять одни токены на другие.

Эта ликвидность называется виртуальной, потому что де-факто далеко не вся эта ликвидность участвует в операции обмена токенов. Как уже было сказано выше, позиции провайдера ликвидности могут быть активными, и неактивными соответственно. Разумеется, не вся ликвидность, находящаяся в пуле, будет активной в произвольный момент, особенно в волатильных парах.

Ликвидность, которая в момент исполнения сделки обмена токенов была активной, составляет реальную ликвидность $(x_{real}, y{real}$. 

\begin{equation}
L = \sqrt{x_v \cdot y_v} = \sqrt{x_r \cdot y_r} \cdot \frac{1}{\sqrt{P} - \sqrt{P_a}} \cdot \frac{1}{\frac{1}{\sqrt{P}} - \frac{1}{\sqrt{P_b}}}
\label{eq:liquidity_virtual}
\end{equation}
\section{Торговое множество и достижимость сделки}

Следующее развитие мысли -- введение понятия торгового множества $T$, которое представляет собой множество сделок, доступных в конкретный момент времени (точнее, для конкретного набора резервов токенов $\mathbf{R}$)

\begin{equation}
T(\mathbf{R}) = \{(\delta, \lambda) | \varphi(\mathbf{R}, \delta', \lambda') = 0\}
\label{eq:trading_set}
\end{equation}

где сделка $(\delta', \lambda')$ -- произвольная сделка, которая "выгоднее" для пользователя, чем сделка $(\delta, \lambda)$, которая попадает в множество доступных сделок $T(R)$.
Под "выгодностью" сделки понимается, что $\delta' \leq \delta$, а $\lambda' \geq \lambda$, т.е. в пересчете на одну единицу предоставляемых пользователем токенов $\delta$ пользователь получит больше токенов $\lambda$.

Таким образом, торговое множество состоит из таких сделок, выгоднее которых нет для пользователя в момент времени, когда в пуле ликвидности находятся активы $\mathbf{R}$.
Идея заключается в том, что пользователя можно представить как рационального агента, которые из двух и более сделок будет выбирать ту, которая для него будет наиболее выгодной.

\section{Независимость торговых путей}
Важно сделать несколько упрощений модели работы CFMM, которые сделают моделирование процессов менее громоздкими. Для этого важно ввести понятие торгового пути (trading path). 

Пусть изначально пул CFMM находится в состоянии $\mathbf{R}^0$. 
После первой сделки он станет $\mathbf{R}^1$, после второй -- $\mathbf{R}^2$ и так далее до $\mathbf{R}^m$, где $m \in \mathbb{Z}_{+}$. 

Последовательность $\{\mathbf{R}^{i}\}^{m}_{i = 0}$ является торговым путем, если эти состояния последовательно достигаются после совершения сделок $\{(\delta, \lambda)^{i}\}^{m}_{i = 0}$.
Последовательность сделок $(\delta, \lambda)^i$ является торговой стратегией.

Независимостью торговых путей является свойство CFMM, при котором можно прийти сразу в состояние $\mathbf{R}^m$, совершив одну большую <<суммарную>> сделку, вместо совершения $m$ сделок из стратегии $\{(\delta, \lambda)^{i}\}^{m}_{i = 0}$. 
Формально это выражается таким свойством:

\begin{equation}
\{R^{i+1} = R^{i} + \delta^i - \lambda^i\}, \quad i = 0, \ldots, m \\
\label{eq:trading_path_indenpence_part_1}
\end{equation}

Равняется

\begin{equation}
R + \sum_{i = 0}^{m}{\delta^i} - \sum_{i = 0}^{m}{\lambda^i}
\label{eq:trading_path_independence_part_2}
\end{equation}

Схематично это можно изобразить следующим образом:
\begin{figure}[!h]
  \begin{center}
    \includegraphics[width=\textwidth]{Chapter1-path-independence.pdf}
\caption{Демонстрация независимости торговых путей в CFMM. Торговый путь, выраженный сплошными и пунктирными стрелками приведет в ту же точку $\mathbf{R}^m$, что и торговый путь, состоящий из единственной верхней стрелки, состоящей из точек и пунктира.}
\label{figure:path_independence_demo}
  \end{center}
\end{figure}


\section{Несовершенство торговых путей}
Следующее, расширяющее эту идею понятие -- несовершенство торговых путей (trade path deficiency).
В CFMM произвольный торговый путь является несовершенным, если:

\begin{equation}
\forall \mathbf{R}, \forall \mathbf{R}' \in S(\mathbf{R}) \; \exists S(\mathbf{R'})\: S(\mathbf{R'}) \subseteq S(\mathbf{R})
\label{eq:trading_path_deficiency}
\end{equation}

где $\mathbf{R}$ -- текущее состояние пула CFMM, $S(\mathbf{R})$ -- множество достижимых резервов из состояния $\mathbf{R}$.

Важным для дальнейшей работы является следствие о том, что резервы в пуле, вне зависимости от количества проведенных сделок, будут ограничены снизу:
\begin{equation}
\forall n \in \mathbb{N}: \; \mathbf{1}^T \cdot \mathbf{R^n} \geq \inf \mathbf{1}^T \cdot \mathbf{R^0}
\label{eq:trade_fn_general}
\end{equation}
где $\mathbf{R^n}$ -- состояние пула CFMM после $n$ совершенных сделок. $\mathbf(R^0)$ соответственно означает начальное состояние пула.

\section{Сравнительный анализ Uniswap V2 и Uniswap V3 с точки зрения несовершенства торговых путей}
Важное изменение между Uniswap V2 и Uniswap V3 заключается во внедрении диапазонов действия ликвидности. В Uniswap V2 ликвидность предоставлялась на весь диапазон цены, т.е. на $[0; \infty)$. В Uniswap V3 при предоставлении ликвидности пользователь указывает также диапазон цены, в котором его ликвидность будет функционировать. В данной работе этот вопрос будет рассмотрен более детально в следующей главе, а пока следует указать, что формальное определение, данное в формуле (\ref{eq:trading_path_deficiency}), на практике означает, что последовательность сделок теперь имеет значение.

В Uniswap V2, где функция торговли является <<чистым>> CPMM, маркетмейкером постоянного произведения, конечное состояние пула $R^m$ зависит только от чистого потока токенов, как показано в (\ref{eq:trading_path_independence_part_2}). Торговые пути независимы в Uniswap V2.
В Uniswap V3, т.е. с концентрированной ликвидностью, конечное состояние пула $\mathbf{R}^m$ зависит от конкретной последовательности цен, по которым проходили сделки. Это связано с тем, что активные резервы в пуле V3 определяются текущей ценой. Когда цена выходит за пределы диапазона $[P_a, P_b]$ конкретной позиции ликвидности, эта позиция <<деактивируется>> и перестает участвовать в торговле. Таким образом, множество достижимых состояний $S(\mathbf{R})$ динамически меняется с каждым ценовым движением. Свойство \eqref{eq:trading_path_deficiency} как раз говорит о том, что множество достижимых состояний может только сужаться.

Пока цена находится в диапазоне цены, заданном при создании позиции ликвидности, Uniswap V3 ведет себя почти также, как и Uniswap V2. Как только сделка (или серия сделок) сдвигает цену так, что\ покидает текущий активный диапазон $[P_a, P_b]$ и пересекает границу $P_a$ или $P_b$, применимо свойство несовершенства торговых путей.

При этом обычно, когда позиция ликвидности деактивируется, это означает потенциально меньший диапазон возможных состояний пула, особенно когда цена в процессе исполнения сделки <<уходит>> от рыночной (поскольку большинство остальных пользователей скорее всего предоставили ликвидность в окрестность точки рыночной цены), тем самым выполняется свойство \ref{eq:trading_path_deficiency}: $S(\mathbf{R'}) \subseteq S(\mathbf{R})$

\section{Цели и задачи магистерской работы}


\textbf{Цель работы} -- not done

\textbf{Задачи работы:}
\begin{enumerate}
\item not done.
\item not done.
\item not done.
\item not done.
\end{enumerate}
