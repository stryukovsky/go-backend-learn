\newpage
\begin{center}
	\textbf{\large 4. СТРАТЕГИИ ПРЕДОСТАВЛЕНИЯ ЛИКВИДНОСТИ В UNISWAP V3}
\end{center}
\refstepcounter{chapter}
\addcontentsline{toc}{chapter}{4. СТРАТЕГИИ ПРЕДОСТАВЛЕНИЯ ЛИКВИДНОСТИ В UNISWAP V3}

\section{Введение}


\subsection{Накопление комиссий}

Комиссии $F$ накапливаются пропорционально доле ликвидности позиции в общей активной ликвидности и времени нахождения в активном состоянии:

\begin{equation}
	F = \int_{t_0}^{t_1} \frac{L_{position}}{L_{active}(t)} \cdot fee \cdot V(t) \, dt,
	\label{eq:fees_income_formula}
\end{equation}
где:
\begin{itemize}
	\item $L_{\text{position}}$ — ликвидность отдельной позиции поставщика ликвидности;
	\item $L_{\text{active}}(t)$ — суммарная активная ликвидность во всём пуле в момент времени $t$;
	\item $fee$ — уровень торговой комиссии (зависит от пула)
	\item $V(t)$ — мгновенный объём торгов в пуле в момент времени $t$.
\end{itemize}

Из формулы~\eqref{eq:fees_income_formula} непосредственно следуют три ключевых стратегических императива, определяющих поведение рациональных (в частности, институциональных) поставщиков ликвидности:

\begin{enumerate}
	\item Максимизация доли в активной ликвидности через концентрацию.
	      Поскольку доля $\frac{L_{\text{position}}}{L_{\text{active}}(t)}$ прямо пропорциональна доходу, институционалы стремятся размещать ликвидность в узких ценовых диапазонах. Это позволяет захватывать значительную часть торгового объёма при относительно небольшом капитале. Однако такая стратегия требует постоянного мониторинга цены и частого ребалансирования позиций, что технически доступно преимущественно институциональным участникам с алгоритмическими системами управления.

	\item Выбор уровня комиссии через выбор типа пула.
	      Уровень комиссии $\text{fee}$ фиксирован на уровне контракта пула и не может быть изменён LP после входа. Следовательно, стратегический выбор сводится к выбору между пулами со стейблкоинами (где ниже комиссия и ниже непостоянные потери) или где хотя бы один токен не является стейблкоином (где возникают более ощутимые непостоянные потери из-за движения цены, но и комиссия поставщика ликвидности выше)
	\item Фокус на пулах с максимальным объёмом торгов.
	      Доход пропорционален $V(t)$ — мгновенному объёму торгов. Возникает гипотеза о том, что институционалы целенаправленно концентрируют капитал в наиболее ликвидных пулах ($ETH/USDC$, $WBTC/ETH$, $USDC/USDT$), где $V(t)$ максимален, даже если это означает конкуренцию с другими крупными LP. Для них важна не абсолютная доля, а абсолютный размер комиссий, который растёт с ростом $V(t)$.
\end{enumerate}

Практическая часть данной работы ставит своей целью проверка данных гипотез на практических данных. 
