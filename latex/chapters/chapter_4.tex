\newpage
\begin{center}
	\textbf{\large 4. СТРАТЕГИИ ПРЕДОСТАВЛЕНИЯ ЛИКВИДНОСТИ В UNISWAP V3}
\end{center}
\refstepcounter{chapter}
\addcontentsline{toc}{chapter}{4. СТРАТЕГИИ ПРЕДОСТАВЛЕНИЯ ЛИКВИДНОСТИ В UNISWAP V3}

\section{Введение}


\subsection{Накопление комиссий}

Комиссии $F$ накапливаются пропорционально доле ликвидности позиции в общей активной ликвидности и времени нахождения в активном состоянии:

\begin{equation}
	F = \int_{t_0}^{t_1} \frac{L_{position}}{L_{active}(t)} \cdot fee \cdot V(t) \, dt,
	\label{eq:fees_income_formula}
\end{equation}
где:
\begin{itemize}
	\item $L_{\text{position}}$ — ликвидность отдельной позиции поставщика ликвидности;
	\item $L_{\text{active}}(t)$ — суммарная активная ликвидность во всём пуле в момент времени $t$;
	\item $fee$ — уровень торговой комиссии (зависит от пула)
	\item $V(t)$ — мгновенный объём торгов в пуле в момент времени $t$.
\end{itemize}
