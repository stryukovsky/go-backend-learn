\newpage
\begin{center}
	\textbf{\large 5. МЕТОДОЛОГИЯ ОПРЕДЕЛЕНИЯ ИНСТИТУЦИОНАЛЬНЫХ УЧАСТНИКОВ РЫНКА ПРЕДОСТАВЛЕНИЯ ЛИКВИДНОСТИ}
\end{center}
\refstepcounter{chapter}
\addcontentsline{toc}{chapter}{4. МЕТОДОЛОГИЯ ОПРЕДЕЛЕНИЯ ИНСТИТУЦИОНАЛЬНЫХ УЧАСТНИКОВ РЫНКА ПРЕДОСТАВЛЕНИЯ ЛИКВИДНОСТИ}

\section{Идея исходной статьи}
Изначально сотрудники центра блокчейн СБЕРа посчитали интересной статью
\cite{bis_1227}, в которой подтверждается тезис о том, что несмотря на
технические возможности децентрализации рынка предоставления ликвидности,
фактически этот рынок остается во влиянии институциональных участников.

В статье авторы приводят следующую методологию классификации участников рынка
предоставления ликвидности UniswapV3 как институциональных (в тексте статьи
такие участники называются sophisticated):

\begin{enumerate}
	\item максимальный объём ликвидности, предоставленной
	      участником (в долларах США). Участник считается институциональным, если его
	      значение превышает 95-й процентиль распределения этого показателя по всем
	      участникам;
	\item все участники, которые хотя бы раз создавали позицию
	      стоимостью не менее 1 млн долларов США, классифицируются как институциональные;
	\item общее количество созданных позиций ликвидности. Участник считается
	      институциональным, если его значение превышает 95-й процентиль распределения
	      этого показателя по всем участникам;
	\item количество различных пулов, в которых участник предоставлял ликвидность.
	      Участник считается институциональным, если его значение превышает 95-й
	      процентиль распределения этого показателя по всем участникам;
	\item общее число транзакций, связанных с предоставлением ликвидности. Участник
	      считается институциональным, если его значение превышает 95-й процентиль
	      распределения этого показателя по всем участникам.

\end{enumerate}

Эта методология сотрудниками была принята как подходящая для подобных
исследований централизованности рынка предоставления ликвидности.

В свою же очередь данная работа развивает некоторые идеи этой работы, а именно:

\begin{enumerate}
	\item В исходной работе доказывается факт о том, что
	      низковолатильные пулы привлекают большее внимание институциональных участников.
	      Какая доля в таких пулах приходится на арбитражные сделки?
	\item Насколько
	      быстро институциональные участники реагируют на крупные экономические события,
	      которые в течение короткого времени влияют на стоимость стейблокинов?
	\item На
	      какие источники данных опираются институциональные участники при задании
	      ценового диапазона позиции ликвидности?
	\item Каким образом институциональные
	      участники минимизируют (и минимизируют ли) непостоянные потери (Impermanent
	      Loss, IL)?
	\item Насколько высок порог входа для обычного пользователя, чтобы
	      он мог копировать стратегии, доступные институциональным участникам?
\end{enumerate}


