\newpage
\begin{center}
	\textbf{\large 5. МЕТОДОЛОГИЯ ОПРЕДЕЛЕНИЯ ИНСТИТУЦИОНАЛЬНЫХ УЧАСТНИКОВ РЫНКА ПРЕДОСТАВЛЕНИЯ ЛИКВИДНОСТИ}
\end{center}
\refstepcounter{chapter}
\addcontentsline{toc}{chapter}{4. МЕТОДОЛОГИЯ ОПРЕДЕЛЕНИЯ ИНСТИТУЦИОНАЛЬНЫХ УЧАСТНИКОВ РЫНКА ПРЕДОСТАВЛЕНИЯ ЛИКВИДНОСТИ}

\section{Идея исходной статьи}
Изначально сотрудники центра блокчейн СБЕРа посчитали интересной статью \cite{bis_1227}, в которой подтверждается тезис о том, что несмотря на технические возможности децентрализации рынка предоставления ликвидности, фактически этот рынок остается во влиянии институциональных участников.

В статье авторы приводят следующую методологию классификации участников рынка предоставления ликвидности UniswapV3 как институциональных (в тексте статьи такие участники называются sophisticated):

\begin{enumerate}
	\item максимальный объём ликвидности, предоставленной участником (в долларах США). Участник считается институциональным, если его значение превышает 95-й процентиль распределения этого показателя по всем участникам;
	\item все участники, которые хотя бы раз создавали позицию стоимостью не менее 1 млн долларов США, классифицируются как институциональные;
	\item общее количество созданных позиций ликвидности. Участник считается институциональным, если его значение превышает 95-й процентиль распределения этого показателя по всем участникам;

	\item количество различных пулов, в которых участник предоставлял ликвидность. Участник считается институциональным, если его значение превышает 95-й процентиль распределения этого показателя по всем участникам;

	\item общее число транзакций, связанных с предоставлением ликвидности. Участник считается институциональным, если его значение превышает 95-й процентиль распределения этого показателя по всем участникам.

\end{enumerate}

Эта методология сотрудниками была принята как подходящая для подобных исследований централизованности рынка предоставления ликвидности. 


