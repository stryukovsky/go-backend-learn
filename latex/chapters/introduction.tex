\newpage
\begin{center}
  \textbf{\large АННОТАЦИЯ}
\end{center}

За последний год влияние институциональных игроков на блокчейн-экосистему достигло беспрецедентных масштабов, что значительно повлияло на ландшафт цифровых активов. Массовое внедрение биткоин-ETF, стратегические инвестиции крупнейших финансовых институтов и корпораций, а также растущее признание криптовалют в качестве легитимного класса активов ознаменовали переломный момент в истории индустрии. Институциональное участие не только обеспечило существенный приток капитала и ликвидности на рынок, но и способствовало развитию регуляторных рамок, повышению доверия со стороны традиционных инвесторов и ускорению технологической зрелости блокчейн-инфраструктуры. Данная работа анализирует ключевые аспекты институционального влияния на один из основных рынков в DeFi  --- рынок предоставления ликвидности. В данной работе будет рассмотрен протокол Uniswap V3, являющийся одним из классических протоколов децентрализованных обменников.

\onehalfspacing
\setcounter{page}{2}

\newpage
\renewcommand{\contentsname}{\centerline{\large СОДЕРЖАНИЕ}}
\tableofcontents

\newpage
\begin{center}
  \textbf{\large ВВЕДЕНИЕ}
\end{center}
\addcontentsline{toc}{chapter}{ВВЕДЕНИЕ}


\textbf{Актуальность}


Но остаются открытыми следующие важные вопросы:
\begin{enumerate}
\item Какие стратегии выбирают институциональные участники рынка предоставления ликвидности, чтобы максимизировать свою прибыль?
\item Какое влияние они оказывают на рынок предоставления ликвидности вцелом?
\end{enumerate}

\newpage

\textbf{Цель магистерской квалификационной работы} -- Изучение влияния институциональных участников рынка предоставления ликвидности, и следующая из этого проверка гипотезы о том, что институциональные игроки почти всегда опережают обычных пользователей в доходах на рынке предоставления ликвидности протокола UniswapV3. Принято считать, что институциональные участники активнее управляют своей ликвидностью, размещают ликвидность в более узких ценовых диапазонах, они лучше знают движения стоимости тех или иных активов, что позволяет им точнее решать задачу арбитража, что также влияет на доход поставщиков ликвидности, они работают со значительными денежными объемами, что позволяет им более агрессивно вести себя в пулах ликвидности со стейблкоинами. Эти предпосылки теоретически дают институциональным участникам преимущество. Эти гипотезы а также статьи, в которых они были выдвинуты, будут рассмотрены подробнее в данной работе. В работе сперва будет приведено теоретическое обоснование тех или иных преимуществ институциональных участников, а затем на собранных данных из блокчейна Ethereum будет сформирован датасет и проведен анализ данных. Полученная картина позволит оценить влияние институциональных участников рынка предоставления ликвидности протокола UniswapV3.


\textbf{Задачи магистерской квалификационной работы:}
\begin{enumerate}
\item Разработка инструментария сбора массива данных о действиях участников протокола Uniswap V3;
\item Разработка инструментов анализа собранных данных с целью определения, кто из участников являются институциональными, а кто --- обычными пользователями;
\item Разработка инструментов анализа доходности действий институциональных и обычных участников;
\item Осуществление анализа деятельности участников блокчейн-протокола Uniswap V3. 
\end{enumerate}


\textbf{Научной новизной обладают следующие результаты магистерской
  квалификационной работы:}
\begin{enumerate}
\item Изучены стратегии предоставления ликвидности институциональными участниками рынка предоставления ликвидности
\item Проведен анализ влияния институциональных участников на весь рынок предоставления ликвидности
\item Проведена количественная оценка влияния стратегий управления ликвидностью на доход для институциональных участников и для рыночных участников
\item Анализ неравенства в DeFi как новой формы рыночной асимметрии
\end{enumerate}
