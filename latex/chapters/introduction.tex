\newpage
\begin{center}
	\textbf{\large АННОТАЦИЯ}
\end{center}

За последние два года участие институциональных участников в экосистеме блокчейн увеличилось. Массовое внедрение биткоин-ETF, стратегические инвестиции крупнейших финансовых институтов и корпораций, а также растущее признание криптовалют в качестве легитимного класса активов ознаменовали переломный момент в истории индустрии.

Например, согласно \cite{coinshares_btc_etf}, за 2025 год объем участия институциональных участников в Bitcoin ETF (англ. Exchange Trading Fund, биржевой паевый фонд) вырос с 12 миллиардов долларов США до 37 миллиардов долларов, т.е. более чем на 200\%.

Институциональное участие не только обеспечило существенный приток капитала и ликвидности на рынок, но и способствовало развитию регуляторных рамок, повышению доверия со стороны традиционных инвесторов и ускорению технологической зрелости блокчейн-инфраструктуры.

Данная работа анализирует ключевые аспекты институционального влияния на один из основных рынков в DeFi  --- рынок предоставления ликвидности. В данной работе будет рассмотрен протокол Uniswap V3, являющийся одним из классических протоколов децентрализованных обменников.

\onehalfspacing
\setcounter{page}{2}

\newpage
\renewcommand{\contentsname}{\centerline{\large СОДЕРЖАНИЕ}}
\tableofcontents

\newpage
\begin{center}
	\textbf{\large ВВЕДЕНИЕ}
\end{center}
\addcontentsline{toc}{chapter}{ВВЕДЕНИЕ}


\textbf{Актуальность}

Присутствие институциональных участников за последние два года становится все более заметным. Такие участники обладают рядом конкурентных преимуществ: доступом к высокочастотным данным, алгоритмическим системам управления позициями, значительными объёмами капитала, а также возможностью использовать арбитражные и MEV-стратегии для максимизации дохода. В то же время розничные пользователи, как правило, используют статичные или полуавтоматические стратегии, что делает их уязвимыми к непостоянным потерям и снижению эффективной доходности.

Большинство существующих работ фокусируются либо на технической архитектуре протокола, либо на общих показателях эффективности рынка, игнорируя гетерогенность участников. Между тем понимание того, насколько институциональные игроки <<опережают>> обычных пользователей, имеет важные последствия как для теории рыночного равновесия в DeFi, так и для практики регулирования, проектирования протоколов и защиты интересов розничных инвесторов.

Актуальность настоящего исследования обусловлена следующими факторами:

\begin{enumerate}
	\item протокол Uniswap V3 в экосистеме Ethereum зарекомендовал себя как стабильный протокол децентрализованной биржи. Согласно \cite{defillama_dex_metrics}, на начало 2026 года Uniswap остаётся доминирующим DEX по объёму торгов и TVL;
	\item увеличение активности институциональных участников в DeFi: По данным \cite{mexc_eth_reserves}, доля транзакций, связанных с институциональными адресами, в сегменте ликвидности продолжает расти, особенно в пулах стейблкоинов и ETH-парах.
	\item Отсутствие прозрачности в распределении выгод: Хотя DeFi декларирует принцип <<финансовой демократии>>, на практике формируются новые формы неравенства, обусловленные технологическим и информационным разрывом.
	\item Потребность регуляторов в анализе рисков для обычных, <<розничных>> участников подобных протоколов.
\end{enumerate}

\newpage

\textbf{Цель магистерской квалификационной работы} -- изучение влияния институциональных участников рынка предоставления ликвидности, и следующая из этого проверка гипотезы о том, что институциональные игроки почти всегда опережают обычных пользователей в доходах на рынке предоставления ликвидности протокола UniswapV3.

\textbf{Задачи магистерской квалификационной работы:}
\begin{enumerate}
	\item Разработка инструментария сбора массива данных о действиях участников протокола Uniswap V3;
	\item Разработка инструментов анализа собранных данных с целью определения, кто из участников являются институциональными, а кто -- обычными пользователями;
	\item Разработка инструментов анализа доходности действий институциональных и обычных участников;
	\item Осуществление анализа деятельности участников блокчейн-протокола Uniswap V3.
\end{enumerate}


\textbf{Научной новизной обладают следующие результаты магистерской
	квалификационной работы:}
\begin{enumerate}
	\item Изучены стратегии предоставления ликвидности институциональными участниками рынка предоставления ликвидности
	\item Проведен анализ влияния институциональных участников на весь рынок предоставления ликвидности
	\item Проведена количественная оценка влияния стратегий управления ликвидностью на доход для институциональных участников и для рыночных участников
	\item Анализ неравенства в DeFi как новой формы рыночной асимметрии
\end{enumerate}
