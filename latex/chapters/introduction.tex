\newpage
\begin{center}
  \textbf{\large АННОТАЦИЯ}
\end{center}

За последний год влияние институциональных игроков на блокчейн-экосистему достигло беспрецедентных масштабов, что значительно повлияло на ландшафт цифровых активов. Массовое внедрение биткоин-ETF, стратегические инвестиции крупнейших финансовых институтов и корпораций, а также растущее признание криптовалют в качестве легитимного класса активов ознаменовали переломный момент в истории индустрии. Институциональное участие не только обеспечило существенный приток капитала и ликвидности на рынок, но и способствовало развитию регуляторных рамок, повышению доверия со стороны традиционных инвесторов и ускорению технологической зрелости блокчейн-инфраструктуры. Данная работа анализирует ключевые аспекты институционального влияния на один из основных рынков в DeFi  --- рынок предоставления ликвидности. В данной работе будет рассмотрен протокол Uniswap V3, являющийся одним из классических протоколов децентрализованных обменников.

\onehalfspacing
\setcounter{page}{2}

\newpage
\renewcommand{\contentsname}{\centerline{\large СОДЕРЖАНИЕ}}
\tableofcontents

\newpage
\begin{center}
  \textbf{\large ВВЕДЕНИЕ}
\end{center}
\addcontentsline{toc}{chapter}{ВВЕДЕНИЕ}


\textbf{Актуальность}


Но остаются открытыми следующие важные вопросы:
\begin{enumerate}
\item Какое влияние имеет потенциал взаимодействия между частицами на температурную зависимость коэффициента диффузии;
\item Насколько важны корреляции между спектрами возбуждений и транспортными свойствами.
\end{enumerate}

\newpage

\textbf{Цель магистерской квалификационной работы} -- установить связь \\ дальнодействия притяжения потенциала взаимодействия и спектров возбуждений с транспортными свойствами жидкостей, а также выявить влияние дальнодействия притяжения на скорость нуклеации.

\textbf{Задачи магистерской квалификационной работы:}
\begin{enumerate}
\item Разработка инструментария сбора массива данных о действиях участников протокола Uniswap V3;
\item Разработка инструментов анализа собранных данных с целью определения, кто из участников являются институциональными, а кто --- обычными пользователями;
\item Разработка инструментов анализа доходности действий институциональных и обычных участников;
\item Осуществление анализа деятельности участников блокчейн-протокола Uniswap V3. 
\end{enumerate}


\textbf{Научной новизной обладают следующие результаты магистерской
  квалификационной работы:}
\begin{enumerate}
\item Разработан инструмент сбора и хранения данных об
\item При увеличении дальнодействия потенциала увеличивается отношение температур критической к тройной точке.
  Кроме того, при этом уменьшается наклон температурной зависимости подвижности.
\item Отклонение подвижности от линейной зависимости при высоких температурах коррелирует с переходом спектров возбуждений от осцилирующего к монотонному виду.
\end{enumerate}

%
% \textbf{Апробация} основных результатов магистерской квалификационной работы проводилась на следующих конференциях:
% \begin{enumerate}
% \item XX Школа-конференция молодых ученых <<Проблемы физики твердого тела и высоких давлений>>, Сочи, 16-26 сентября 2021г.
% \item Современные тенденции развития функциональных материалов, Сочи, 11-14 ноября 2021г.
% \item Dynamic phenomena workshop 2022.
% \end{enumerate}
%
%
% \textbf{Публикации:}
% \begin{enumerate}
% \item Kryuchkov, N. P., Dmitryuk, N. A., Li, W., Ovcharov, P. V., Han, Y., Sapelkin, A. V., and Yurchenko, S. O. (2021). \\ Mean-field model of melting in superheated crystals based on a single \\ experimentally measurable order parameter. Scientific reports, 11(1), 1-15.
% \item Yakovlev, E. V., Kryuchkov, N. P., Korsakova, S. A., Dmitryuk, N. A., Ovcharov, P. V., Andronic, M. M., ... and Yurchenko, S. O. (2022). 2D colloids in rotating electric fields: A laboratory of strong tunable three-body interactions. Journal of Colloid and Interface Science, 608, 564-574.
% \item Tsiok, E. N., Fomin, Y. D., Gaiduk, E. A., Tareyeva, E. E., Ryzhov, V. N., Libet, P. A., ... Yurchenko, S. O. (2022). The role of attraction in the phase diagrams and melting scenarios of generalized 2D Lennard-Jones systems. The Journal of Chemical Physics, 156(11), 114703.
% \end{enumerate}
